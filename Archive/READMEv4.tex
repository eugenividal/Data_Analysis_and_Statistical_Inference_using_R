\documentclass[]{article}
\usepackage{lmodern}
\usepackage{amssymb,amsmath}
\usepackage{ifxetex,ifluatex}
\usepackage{fixltx2e} % provides \textsubscript
\ifnum 0\ifxetex 1\fi\ifluatex 1\fi=0 % if pdftex
  \usepackage[T1]{fontenc}
  \usepackage[utf8]{inputenc}
\else % if luatex or xelatex
  \ifxetex
    \usepackage{mathspec}
  \else
    \usepackage{fontspec}
  \fi
  \defaultfontfeatures{Ligatures=TeX,Scale=MatchLowercase}
\fi
% use upquote if available, for straight quotes in verbatim environments
\IfFileExists{upquote.sty}{\usepackage{upquote}}{}
% use microtype if available
\IfFileExists{microtype.sty}{%
\usepackage{microtype}
\UseMicrotypeSet[protrusion]{basicmath} % disable protrusion for tt fonts
}{}
\usepackage[margin=1in]{geometry}
\usepackage{hyperref}
\hypersetup{unicode=true,
            pdftitle={Road Traffic Accidents Data Analysis using R},
            pdfauthor={Student ID: 201081646},
            pdfborder={0 0 0},
            breaklinks=true}
\urlstyle{same}  % don't use monospace font for urls
\usepackage{color}
\usepackage{fancyvrb}
\newcommand{\VerbBar}{|}
\newcommand{\VERB}{\Verb[commandchars=\\\{\}]}
\DefineVerbatimEnvironment{Highlighting}{Verbatim}{commandchars=\\\{\}}
% Add ',fontsize=\small' for more characters per line
\usepackage{framed}
\definecolor{shadecolor}{RGB}{248,248,248}
\newenvironment{Shaded}{\begin{snugshade}}{\end{snugshade}}
\newcommand{\KeywordTok}[1]{\textcolor[rgb]{0.13,0.29,0.53}{\textbf{{#1}}}}
\newcommand{\DataTypeTok}[1]{\textcolor[rgb]{0.13,0.29,0.53}{{#1}}}
\newcommand{\DecValTok}[1]{\textcolor[rgb]{0.00,0.00,0.81}{{#1}}}
\newcommand{\BaseNTok}[1]{\textcolor[rgb]{0.00,0.00,0.81}{{#1}}}
\newcommand{\FloatTok}[1]{\textcolor[rgb]{0.00,0.00,0.81}{{#1}}}
\newcommand{\ConstantTok}[1]{\textcolor[rgb]{0.00,0.00,0.00}{{#1}}}
\newcommand{\CharTok}[1]{\textcolor[rgb]{0.31,0.60,0.02}{{#1}}}
\newcommand{\SpecialCharTok}[1]{\textcolor[rgb]{0.00,0.00,0.00}{{#1}}}
\newcommand{\StringTok}[1]{\textcolor[rgb]{0.31,0.60,0.02}{{#1}}}
\newcommand{\VerbatimStringTok}[1]{\textcolor[rgb]{0.31,0.60,0.02}{{#1}}}
\newcommand{\SpecialStringTok}[1]{\textcolor[rgb]{0.31,0.60,0.02}{{#1}}}
\newcommand{\ImportTok}[1]{{#1}}
\newcommand{\CommentTok}[1]{\textcolor[rgb]{0.56,0.35,0.01}{\textit{{#1}}}}
\newcommand{\DocumentationTok}[1]{\textcolor[rgb]{0.56,0.35,0.01}{\textbf{\textit{{#1}}}}}
\newcommand{\AnnotationTok}[1]{\textcolor[rgb]{0.56,0.35,0.01}{\textbf{\textit{{#1}}}}}
\newcommand{\CommentVarTok}[1]{\textcolor[rgb]{0.56,0.35,0.01}{\textbf{\textit{{#1}}}}}
\newcommand{\OtherTok}[1]{\textcolor[rgb]{0.56,0.35,0.01}{{#1}}}
\newcommand{\FunctionTok}[1]{\textcolor[rgb]{0.00,0.00,0.00}{{#1}}}
\newcommand{\VariableTok}[1]{\textcolor[rgb]{0.00,0.00,0.00}{{#1}}}
\newcommand{\ControlFlowTok}[1]{\textcolor[rgb]{0.13,0.29,0.53}{\textbf{{#1}}}}
\newcommand{\OperatorTok}[1]{\textcolor[rgb]{0.81,0.36,0.00}{\textbf{{#1}}}}
\newcommand{\BuiltInTok}[1]{{#1}}
\newcommand{\ExtensionTok}[1]{{#1}}
\newcommand{\PreprocessorTok}[1]{\textcolor[rgb]{0.56,0.35,0.01}{\textit{{#1}}}}
\newcommand{\AttributeTok}[1]{\textcolor[rgb]{0.77,0.63,0.00}{{#1}}}
\newcommand{\RegionMarkerTok}[1]{{#1}}
\newcommand{\InformationTok}[1]{\textcolor[rgb]{0.56,0.35,0.01}{\textbf{\textit{{#1}}}}}
\newcommand{\WarningTok}[1]{\textcolor[rgb]{0.56,0.35,0.01}{\textbf{\textit{{#1}}}}}
\newcommand{\AlertTok}[1]{\textcolor[rgb]{0.94,0.16,0.16}{{#1}}}
\newcommand{\ErrorTok}[1]{\textcolor[rgb]{0.64,0.00,0.00}{\textbf{{#1}}}}
\newcommand{\NormalTok}[1]{{#1}}
\usepackage{graphicx,grffile}
\makeatletter
\def\maxwidth{\ifdim\Gin@nat@width>\linewidth\linewidth\else\Gin@nat@width\fi}
\def\maxheight{\ifdim\Gin@nat@height>\textheight\textheight\else\Gin@nat@height\fi}
\makeatother
% Scale images if necessary, so that they will not overflow the page
% margins by default, and it is still possible to overwrite the defaults
% using explicit options in \includegraphics[width, height, ...]{}
\setkeys{Gin}{width=\maxwidth,height=\maxheight,keepaspectratio}
\IfFileExists{parskip.sty}{%
\usepackage{parskip}
}{% else
\setlength{\parindent}{0pt}
\setlength{\parskip}{6pt plus 2pt minus 1pt}
}
\setlength{\emergencystretch}{3em}  % prevent overfull lines
\providecommand{\tightlist}{%
  \setlength{\itemsep}{0pt}\setlength{\parskip}{0pt}}
\setcounter{secnumdepth}{5}
% Redefines (sub)paragraphs to behave more like sections
\ifx\paragraph\undefined\else
\let\oldparagraph\paragraph
\renewcommand{\paragraph}[1]{\oldparagraph{#1}\mbox{}}
\fi
\ifx\subparagraph\undefined\else
\let\oldsubparagraph\subparagraph
\renewcommand{\subparagraph}[1]{\oldsubparagraph{#1}\mbox{}}
\fi

%%% Use protect on footnotes to avoid problems with footnotes in titles
\let\rmarkdownfootnote\footnote%
\def\footnote{\protect\rmarkdownfootnote}

%%% Change title format to be more compact
\usepackage{titling}

% Create subtitle command for use in maketitle
\newcommand{\subtitle}[1]{
  \posttitle{
    \begin{center}\large#1\end{center}
    }
}

\setlength{\droptitle}{-2em}

  \title{Road Traffic Accidents Data Analysis using R}
    \pretitle{\vspace{\droptitle}\centering\huge}
  \posttitle{\par}
    \author{Student ID: 201081646}
    \preauthor{\centering\large\emph}
  \postauthor{\par}
    \date{}
    \predate{}\postdate{}
  
\usepackage{float} \floatplacement{figure}{H}

\begin{document}
\maketitle

\section{Introduction}\label{introduction}

This report is the second assessment of the \textbf{MATH5741M
Statistical Theory and Methods} module. Its aim is to answer three
questions through inferential statistical analysis regarding a road
traffic accidents dataset from 2005 collected by the UK Department for
Transport.

All the analysis has been done using \textbf{R} (programming language)
and is code reproducible. To see the complete \textbf{R} coding process
visit
\url{https://github.com/eugenividal/Road-Traffic-Accidents-Data-Analysis}

\section{Results}\label{results}

\subsection{Question 1}\label{question-1}

In this question, first, we are asked to draw a boxplot to compare the
number of vehicles involved in urban areas with the number involved in
rural areas. Secondly, we have to carry out a test to investigate
whether the average number of vehicles in an accident differs per type
of area.

To plot the graph, we first remove the unallocated values.

\begin{figure}[H]

{\centering \includegraphics{READMEv4_files/figure-latex/fig-1} 

}

\caption{Number of vehicles involved grouped by type of area}\label{fig:fig}
\end{figure}

Apart from the fact that rural areas have more ouliers than urban areas,
we cannot appreciate differences between them. Both boxes are very
similar and in both cases the median and upper quartile seem to be
coincident.

From the histograms H1 we can see that the data is very skewed to the
right. The common and convenient assumption in accident count analysis
is that accidents are Poisson-distributed.

To improve this, we take the log10 of the Number of Vehicles variable
and plot new histograms, H2. The data is not symetric yet, but it has
improved.

\begin{figure}[H]

{\centering \includegraphics{READMEv4_files/figure-latex/fig2-1} 

}

\caption{Histograms}\label{fig:fig2}
\end{figure}

A second boxpot is draw with the transformation.

\begin{figure}[H]

{\centering \includegraphics{READMEv4_files/figure-latex/fig3-1} 

}

\caption{Number of vehicles (log10) involved grouped by type of area}\label{fig:fig3}
\end{figure}

This time the size of the boxes is bigger; however, we cannot be sure
wheter the average number of vehicles involved differs per type of area.

To investigate this we carry out statistical test. The null hypothesis
is both means are equal. The alternative hypothesis that they are
different.

\[H_{0}: \mu_{r} = \mu_{u}\;\;\;vs.\;\;\;H_{1}: \mu_{r} \neq \mu_{u}\]
The summary statistics are:

\[n_{r}=72267\;\;\bar{x_{r}}=0.2389048\;\;s_{r}^{2}=0.1775904\;\;n_{u}=126378\;\;\bar{x_{u}}=0.2305898\;\;s_{u}^{2}=0.1622405\]

We will use a critical region approach with \(\alpha\) = 0.01.

It seems reasonable to assume \(\sigma_{1}^{2}\) = \(\sigma_{2}^{2}\),
so we apply a \textbf{pooled estimate}.

First, we calculate the pooled variance.

\[s_{p}^{2}=\frac{(n_{r}-1)s_{r}^{2}+(n_{u}-1)s_{u}^{2}}{n_{r}+n_{u}-2} = \frac{72266*0.03153835017+126377*0.02632197984}{72267+126378-2}=0.0282197\]

Then, we compute the test statistics.

\[\frac{\bar x_{u}-\bar x_{r}-0}{s_{p}\sqrt{\frac{1}{n_{u}}+\frac{1}{n_{r}}}}=\frac{0.2389048-0.2305898}{0.0001316089}=0.001374182\]

We compare this to the critical point of t-distribution with \(\nu\) =
198643 degrees of freedom, which is
\(t_{198643}\)(0.005)=2.575854\footnote{With this number of degrees of
  freedom we could have also apply normal distribution}. Since
0.001374182 \textless{} 2.575854, we do not reject the null hypotesis
and conclude that \(\mu_{r}\) = \(\mu_{u}\).

\subsection{Question 2}\label{question-2}

In this question, first, we have to investigate whether the frequency of
accidents varies by day of the week using the suitable statistical
hypothesis test.

For this, we apply a \textbf{chi-squared test}. Check example p.91.

Be aware this is a poisson distribution (!). The common and convenient
assumption in accident count analysis is that accidents are
Poisson-distributed.

\url{http://www.jbstatistics.com/chi-square-tests-for-one-way-tables/}

Next, we are required to do the same test using only week-days
(excluding Saturday and Sunday).

\subsection{Question 3}\label{question-3}

Finally, in question 3, we are asked to compute a 95\% confidence
interval for the expected (mean) number of accidents which occur on a
Monday.

For this, we apply a simple t-distribution confidence
interval.\footnote{Again in this case, the sample is big enough (n=52)
  to apply a normal distribution confidence interval}.

\begin{Shaded}
\begin{Highlighting}[]
\CommentTok{# Select data using only week-days}
\NormalTok{xx_mon<-}\StringTok{ }\NormalTok{xx%>%}
\StringTok{  }\KeywordTok{select}\NormalTok{(Day_of_Week, Date, count)%>%}
\StringTok{  }\KeywordTok{filter}\NormalTok{(Day_of_Week==}\StringTok{"Monday"}\NormalTok{)%>%}
\StringTok{  }\KeywordTok{group_by}\NormalTok{(Date)%>%}
\StringTok{  }\KeywordTok{summarise}\NormalTok{(}\DataTypeTok{n =} \KeywordTok{sum}\NormalTok{(count))}
\end{Highlighting}
\end{Shaded}

\begin{Shaded}
\begin{Highlighting}[]
\CommentTok{# Compute the interval}
\KeywordTok{t.test}\NormalTok{(xx_mon$n)$conf.int}
\end{Highlighting}
\end{Shaded}

\begin{verbatim}
## [1] 508.9586 560.7337
## attr(,"conf.level")
## [1] 0.95
\end{verbatim}

Checking assumptions for infarence p.84.

The assumptions in computing this interval are:

\begin{enumerate}
\def\labelenumi{(\arabic{enumi})}
\item
  the sample is random and that
\item
  the data is normally distributed. A histogram or a boxplot can be used
  to visualise whether the data are normally distributed. Q-Q plots.
  There are various formal hypothesis tests that can be used to check
  normality. The test that we will use is the Shapiro-Wilk test.
\end{enumerate}

\begin{figure}[H]

{\centering \includegraphics{READMEv4_files/figure-latex/fig4-1} 

}

\caption{Histogram}\label{fig:fig4}
\end{figure}

\begin{Shaded}
\begin{Highlighting}[]
\KeywordTok{shapiro.test}\NormalTok{(xx_mon$n)}
\end{Highlighting}
\end{Shaded}

\begin{verbatim}
## 
##  Shapiro-Wilk normality test
## 
## data:  xx_mon$n
## W = 0.98537, p-value = 0.7681
\end{verbatim}

Since the p-value here is large (i.e.~greater than 0.05) we accept that
the data come from a normal distribution.

\section{Bibliography}\label{bibliography}

The resources used to carry out this project were:

\begin{itemize}
\tightlist
\item
  Balka, J. 2013. JBStatistics: Making Statistics Make Sense. Available
  from: \url{http://www.jbstatistics.com}.
\item
  Lane, D.M. 2018. Online Statistics Education: An Interactive
  Multimedia Course of Study. Available from:
  \url{http://onlinestatbook.com/}.
\item
  Taylor, C. 2017. MATH5741M: Statistical Theory and Methods. Outline
  Lecture Notes.
\item
  Yau, C. 2018. R tutorial: an R introduction to statistics. Available
  from: \url{http://www.r-tutor.com}
\end{itemize}


\end{document}
