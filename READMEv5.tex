\documentclass[]{article}
\usepackage{lmodern}
\usepackage{amssymb,amsmath}
\usepackage{ifxetex,ifluatex}
\usepackage{fixltx2e} % provides \textsubscript
\ifnum 0\ifxetex 1\fi\ifluatex 1\fi=0 % if pdftex
  \usepackage[T1]{fontenc}
  \usepackage[utf8]{inputenc}
\else % if luatex or xelatex
  \ifxetex
    \usepackage{mathspec}
  \else
    \usepackage{fontspec}
  \fi
  \defaultfontfeatures{Ligatures=TeX,Scale=MatchLowercase}
\fi
% use upquote if available, for straight quotes in verbatim environments
\IfFileExists{upquote.sty}{\usepackage{upquote}}{}
% use microtype if available
\IfFileExists{microtype.sty}{%
\usepackage{microtype}
\UseMicrotypeSet[protrusion]{basicmath} % disable protrusion for tt fonts
}{}
\usepackage[margin=1in]{geometry}
\usepackage{hyperref}
\hypersetup{unicode=true,
            pdftitle={Data Analysis and Statistical Inference using R},
            pdfauthor={Student ID: 201081646},
            pdfborder={0 0 0},
            breaklinks=true}
\urlstyle{same}  % don't use monospace font for urls
\usepackage{longtable,booktabs}
\usepackage{graphicx,grffile}
\makeatletter
\def\maxwidth{\ifdim\Gin@nat@width>\linewidth\linewidth\else\Gin@nat@width\fi}
\def\maxheight{\ifdim\Gin@nat@height>\textheight\textheight\else\Gin@nat@height\fi}
\makeatother
% Scale images if necessary, so that they will not overflow the page
% margins by default, and it is still possible to overwrite the defaults
% using explicit options in \includegraphics[width, height, ...]{}
\setkeys{Gin}{width=\maxwidth,height=\maxheight,keepaspectratio}
\IfFileExists{parskip.sty}{%
\usepackage{parskip}
}{% else
\setlength{\parindent}{0pt}
\setlength{\parskip}{6pt plus 2pt minus 1pt}
}
\setlength{\emergencystretch}{3em}  % prevent overfull lines
\providecommand{\tightlist}{%
  \setlength{\itemsep}{0pt}\setlength{\parskip}{0pt}}
\setcounter{secnumdepth}{5}
% Redefines (sub)paragraphs to behave more like sections
\ifx\paragraph\undefined\else
\let\oldparagraph\paragraph
\renewcommand{\paragraph}[1]{\oldparagraph{#1}\mbox{}}
\fi
\ifx\subparagraph\undefined\else
\let\oldsubparagraph\subparagraph
\renewcommand{\subparagraph}[1]{\oldsubparagraph{#1}\mbox{}}
\fi

%%% Use protect on footnotes to avoid problems with footnotes in titles
\let\rmarkdownfootnote\footnote%
\def\footnote{\protect\rmarkdownfootnote}

%%% Change title format to be more compact
\usepackage{titling}

% Create subtitle command for use in maketitle
\newcommand{\subtitle}[1]{
  \posttitle{
    \begin{center}\large#1\end{center}
    }
}

\setlength{\droptitle}{-2em}

  \title{Data Analysis and Statistical Inference using R}
    \pretitle{\vspace{\droptitle}\centering\huge}
  \posttitle{\par}
    \author{Student ID: 201081646}
    \preauthor{\centering\large\emph}
  \postauthor{\par}
    \date{}
    \predate{}\postdate{}
  
\usepackage{floatrow}
\floatsetup[figure]{capposition=top}

\begin{document}
\maketitle

\section{Introduction}\label{introduction}

This report is the second assessment of the \textbf{MATH5741M
Statistical Theory and Methods} module. Its aim is to answer three
questions through data analysis and statistical inference regarding a
road traffic accidents dataset from 2005 collected by the UK Department
for Transport.

All the analysis has been done using \textbf{R} (programming language)
and is code reproducible. To see the complete \textbf{R} coding process
and outputs visit
\url{https://github.com/eugenividal/Data_Analysis_and_Statistical_Inference_using_R}.

\section{Results}\label{results}

\subsection{Question 1}\label{question-1}

In this question, we are asked to draw a boxplot to compare the number
of vehicles involved (in accidents) in urban areas with the number
involved in rural areas.

First, we prepare the data removing ``Unallocated'' values from the
\texttt{Urban\_or\_Rural\ Area} variable. Then, we plot the graph.

\begin{figure}[H]

{\centering \includegraphics{READMEv5_files/figure-latex/fig-1} 

}

\caption{Number of vehicles involved in accidents grouped by type of area}\label{fig:fig}
\end{figure}

Apart from the fact that rural areas have more outliers than urban
areas, we cannot appreciate the differences between their quantiles.
Both boxes seem identical and the median and upper quartile seem to be
coincident.

This is because the data is not symmetrical. As we can see in histogram
H1 (Figure 2), the data is very skewed to the right. To normalise it and
improve the interpretability or appearance of the boxplot, we transform
the \texttt{Number\_of\_Vehicles} variable in three different ways:
taking the log10, log2 and using the square root (see histograms, H2, H3
and H4 in Figure 2). In these new histograms, the distribution is not
entirely symmetric, but they have improved, particularly those that take
log10 and log2.

\begin{figure}[H]

{\centering \includegraphics{READMEv5_files/figure-latex/fig2-1} 

}

\caption{Data histogram (H1) and transformed data histograms (H2, H3, H4)}\label{fig:fig2}
\end{figure}

We choose the log10 transformation and draw a second boxpot.

\begin{figure}[H]

{\centering \includegraphics{READMEv5_files/figure-latex/fig3-1} 

}

\caption{Number of vehicles (log10) involved in accidents grouped by type of area}\label{fig:fig3}
\end{figure}

This time the size of the boxes is bigger. However, the interpretation
of the graph is still difficult, and we cannot be 100\% sure whether the
average number of vehicles involved in accidents differs per type of
area.

To investigate this, we carry out a statistical test, which is the
second requirement of this question. The null hypothesis is that mean of
vehicles involved in both types of areas is equal. The alternative
hypothesis is that they differ. Denoting the rural areas by subscript
\emph{r} and urban by subscript \emph{u}, we have:

\[H_{0}: \mu_{u} = \mu_{r}\;\;\;vs.\;\;\;H_{1}: \mu_{u} \neq \mu_{r}\]

We will use a critical region approach with \(\alpha\) = 0.01.

The summary statistics are:

\[n_{u}=126378\;\;\bar{x_{u}}=0.2305898\;\;s_{u}^{2}=0.1622405\;\;n_{r}=72267\;\;\bar{x_{r}}=0.2389048\;\;s_{r}^{2}=0.1775904\]

It seems reasonable to assume \(\sigma_{u}^{2}\) =
\(\sigma_{r}^{2}\).\footnote{We can assume equal variances when the
  ratio of max/min is less than 3 or less than 4 for small samples
  (Taylor 2017, 69).}. Consequently, we apply a \textbf{Pooled
estimate}.

\[s_{p}^{2}=\frac{(n_{u}-1)s_{u}^{2}+(n_{r}-1)s_{r}^{2}}{n_{u}+n_{r}-2} = \frac{126377*0.02632197984+72266*0.03153835017}{126378+72267-2}=0.0282197\]

The test statistic is then,

\[\frac{\bar x_{u}-\bar x_{r}-0}{s_{p}\sqrt{\frac{1}{n_{u}}+\frac{1}{n_{r}}}}=\frac{0.2389048-0.2305898}{0.0001316089}=-0.001374182\]

We compare this to the critical point of t-distribution with \(\nu\) =
198,643 degrees of freedom\footnote{With this number of degrees of
  freedom we could have also apply z-statistic and the result would have
  been nearly the same.}, which is \(t_{198643}\)(0.005)=2.575854. Since
\textbar{}0.001374182\textbar{} \textless{} 2.575854, we do not reject
the null hypothesis and conclude that \(\mu_{u}\) = \(\mu_{r}\).

\subsection{Question 2}\label{question-2}

In this question, we have to investigate whether the frequency of
accidents varies by day of the week using a suitable statistical
hypothesis test. For this, we apply a \textbf{Chi-square test} which can
be used to test whether observed data differ significantly from
theoretical expectations (Lane 2018).

The null hypothesis is that the frequency of accidents is evenly
distributed per days of the week (i.e.~the probability of accidents
occurring per each day is 1/7). The alternative hypothesis is that their
frequency differ (i.e.~the probability of accidents occurring per each
day is not 1/7).

\[H_{0}: p=1/7\;\;\;vs.\;\;\;H_{1}:p\neq1/7\;\] \pagebreak

First, we prepare the data, aggregating it by \texttt{Day\_of\_Week}.
Secondly, we create a table with the observed values per day of week,
the expected values and other necessary contributions for the test.

\begin{longtable}[]{@{}lrrrrrrr@{}}
\caption{Observed, expected and contributions to X\^{}2}\tabularnewline
\toprule
week days & Monday & Tuesday & Wednesday & Thursday & Friday & Saturday
& Sunday\tabularnewline
observed & 27812 & 29219 & 30373 & 29738 & 32738 & 26945 &
21910\tabularnewline
expected & 28390.71 & 28390.71 & 28390.71 & 28390.71 & 28390.71 &
28390.71 & 28390.71\tabularnewline
oi - ei & 578.7143 & -828.2857 & -1982.2857 & -1347.2857 & -4347.2857 &
1445.7143 & 6480.7143\tabularnewline
(oi - ei)\^{}2/ei & 11.79647 & 24.16485 & 138.40640 & 63.93565 &
665.67163 & 73.61878 & 1479.34487\tabularnewline
\bottomrule
\end{longtable}

The value of \(\chi ^2\) = 2456.93865. This can be compared to the
\(\chi ^2\) distribution with 7 - 1 = 6 degrees of freedom, giving a
p-value of 2.2e-16. This p-value represents the probability that we are
wrong in the assumption they are basically not equally distributed. So,
we can reject the null hypothesis and say that the frequency of
accidents is not evenly distributed per days of the week.

Next, we are required to do the same test using only week-days
(excluding Saturday and Sunday).

This time the null hypothesis is that the frequency of accidents is
equally distributed per week days (i.e.~the probability of accidents per
each week day is 1/5). The alternative hypothesis that their frequency
differ (i.e.~the probability of accidents per each week day is not 1/5).

\[H_{0}: p=1/5\;\;\;vs.\;\;\;H_{1}:p\neq1/5\;\]

We prepare the data, aggregating it by \texttt{Day\_of\_Week} and
removing Saturday and Sundays. Then, we create a new table with the
summaries from Monday to Friday.

\begin{longtable}[]{@{}lrrrrr@{}}
\caption{Observed, expected and contributions to X\^{}2}\tabularnewline
\toprule
week days & Monday & Tuesday & Wednesday & Thursday &
Friday\tabularnewline
observed & 27812 & 29219 & 30373 & 29738 & 32738\tabularnewline
expected & 29976 & 29976 & 29976 & 29976 & 29976\tabularnewline
oi - ei & 2164 & 757 & -397 & 238 & -2762\tabularnewline
(oi - ei)\^{}2/ei & 156.221511 & 19.116927 & 5.257840 & 1.889645 &
254.491727\tabularnewline
\bottomrule
\end{longtable}

The value of \(\chi ^2\) = 436.9776. This is compared to the \(\chi ^2\)
distribution with 5-1=4 degree of freedom, giving a p-value again of
2.2e-16. So, again we reject the null hypothesis and state that the
frequency of accidents in week days is also unequally distributed.

\subsection{Question 3}\label{question-3}

Finally, in question 3, we are asked to compute a 95\% confidence
interval for the expected (mean) number of accidents which occur on a
Monday.

First, we prepare the data, filtering the accidents occurred on Mondays
and grouping them by date.

Then, we compute the sample mean and variance to be: \(\bar{x}\)=
534.8462 and \(s^2\)= 92.98627 respectively. Since we desire a 95\%
interval, our \(\alpha\)= 0.05. \(n\) = 52. We then find that
\(t_{51}(0.025)\)= 2.007584.

Substituting all these quantities into the form of the confidence
interval, we have the 95\% confidence interval for the expected number
of accidents on a Monday.

\[\left ( \bar{x} -t_{n-1}(\alpha /2)\frac{s}{\sqrt{n}}, \bar{x} +t_{n-1}(\alpha /2)\frac{s}{\sqrt{n}}\right) = (534.8462-25.88754,\; 534.8462+25.88754) = 508.9586, 560.7337 \]

Computing this interval, we state the assumption that the data are
normally distributed. To check that this assumption is reasonable, we
can draw a histogram.

\begin{figure}[H]

{\centering \includegraphics{READMEv5_files/figure-latex/fig4-1} 

}

\caption{Histogram number of accidents which occur on a Monday}\label{fig:fig4}
\end{figure}

The graph does not show perfect symmetry, but it is much closer to
normal distribution than the data analysed in the first question.

However, to be more certain, there are various formal hypothesis tests
to check normality that can be used. The one that we will perform here
is the \textbf{Shapiro-Wilk test}, which takes account of the expected
values, but also the correlations between the order statistics (Taylor
2017, 85).

These are the hypothesis,

\[H_{0}: data\;come\;from\;a\;normal\;distribution\;\;\;vs.\;\;\;H_{1}:data\;do\;not\;come\;from\;a\;normal\;distribution\]

We can perform Shapiro-Wilk test of normality using the command
\texttt{shapiro.test(x)} in \textbf{R}. The results are W = 0.98537 and
p-value = 0.7681.

W is the the value of the Shapiro-Wilk statistic. The p-values gives
evidence against the null hypothesis. Since the p-value = 0.7681 is
large (i.e.~greater than 0.05), we accept that the data come from a
normally distributed population.

\section*{References}\label{references}
\addcontentsline{toc}{section}{References}

\hypertarget{refs}{}
\hypertarget{ref-lane_online_2018}{}
Lane, David M. 2018. ``Online Statistics Education: An Interactive
Multimedia Course of Study.'' \url{http://onlinestatbook.com/}.

\hypertarget{ref-taylor_math5741m:_2017}{}
Taylor, Charles. 2017. ``MATH5741M: Statistical Theory and Methods.''
School of Mathematics - University of Leeds.


\end{document}
